\documentclass[a4paper]{article}
\usepackage{ctex}
\usepackage{xeCJK}
\usepackage{amsmath}
\usepackage{amsfonts}
\usepackage{amssymb}
\usepackage{graphicx}
\usepackage{colortbl}
\usepackage{fancyvrb}
\usepackage{longtable}
\usepackage{xcolor}
\usepackage[hidelinks]{hyperref}
\usepackage[affil-it]{authblk}
\usepackage[top = 1.0in, bottom = 1.0in, left = 1.0in, right = 1.0in]{geometry}
\usepackage{amsthm}

\setCJKfamilyfont{kai}{KaiTi_GB2312}
\newcommand{\kai}{\CJKfamily{kai}}

\setCJKfamilyfont{song}{SimSun}
\newcommand{\song}{\CJKfamily{song}}

\newcommand\spc{\vspace{6pt}}
\newcommand{\floor}[1]{\lfloor {#1} \rfloor}
\newcommand{\ceil}[1]{\lceil {#1} \rceil}
\newcommand*\chem[1]{\ensuremath{\mathrm{#1}}}

\newtheorem{theorem}{Theorem}[section]
\newtheorem{lemma}[theorem]{Lemma}
\newtheorem{problem}{例题}

% \kai

\date{\today}
%\date{\yestoday}
\title{$\rm NOI2013$ 解题报告}
\author{$\mathcal Pyh$}

\begin{document}

\maketitle

\kai

\section{向量内积}

考虑如何求出题目中的内积,我们把题目中给出的矩阵与它的转置矩阵乘起来,
然后问题就变成了判断除了对角线之外有没有$0$。

当$k=2$时,注意到如果不是$0$就是$1$,所以我们的问题就转化成了判断这两个矩阵相乘是
否等于一个全$1$的矩阵(对角线上的元素就是每个向量自己的内积,可以预处
理出来),于是可以在等式两边各自左乘一个随机01向量。

等式左边乘出来的复杂度是可以接受的,但是等式右边又需要$n^2$。但我们发
现等式右边基本全是$1$,于是可以预处理出随机的变量的和然后进行计算。

当$k=3$时,注意到$2\bmod -1\pmod{3}$,于是我们可以把求出来的每个结果进
行平方,然后判断是不是全$1$即可。

注意到这样就是$n\times d^2$的矩阵相乘,问题完美解决了。

\section{树的计数}

我们把所有点按照bfs序进行重标号,从小到大考虑每个点,于是我们就在逐层
遍历这棵树。

于是我们计算出每个点的深度比上一个点大$1$的概率,然后加起来就是答案
(期望线性性质)。

注意到,当当前点的dfs序小于上一个点的dfs序时,一定会深度加一。因为同一
层bfs序更大的点一定dfs序也更大。

那么这样子的深度加一是否一定对应方案呢,意思是说假设它的dfs序为$x$,上
一层一定可以找到一个dfs序为$x-1$的点吗?(它必须是当前这一层的开头,那
么一定是父亲的第一个儿子,dfs序一定为父亲的加$1$)

首先,bfs序小于它的点中一定有一个dfs序等于$x-1$的存在,因为不管前面的
决策如何,这一步都必须深度加一,那么因为保证有解,所以一定存在这么一个
dfs序为$x-1$的。

其次,这个点一定在上一层吗?这个倒不一定,但是在前面的决策中,一定会使
得接下来有解,所以我们只要在接下来的分析中,注意保证不会出现这种上一层
不存在的情况即可。

显然这一个深度加一的操作一定不会导致不合法的出现,因为这是必须的操作,
在前面的决策保证接下来的局面有合法解的前提下,一定不会导致不合法的情况
出现。

那么当当前点的dfs序大于上一个点的dfs序时会怎么样呢?

显然若深度要加一,那么至少得保证当前点的dfs序是上一个点的dfs序加一(因
为当前点的dfs序大于上一个点的dfs序,所以一定在上一个点的子树中,而当前
点肯定是上一个点的第一个儿子,所以一定得保证)。

如果当前点的dfs序大于上一个点的dfs序加一,那么就不得深度加一,与上面的
情况同理,这是规定动作,没有变化的可能,所以不会导致后续决策出现无解的情况。

现在的问题就变成了,如果当前点的dfs序等于上一个点的dfs序加一,那么有多
大概率能使深度加一。

如果在这里换层了,那么显然后面的点的dfs序都不得超过当前点的dfs序。并且
因为当前点一定是上一个点的第一个儿子,而上一个点是它们那一层的最后一个
点,所以接下来的点一定都在上一个点的子树中。那么dfs序一定在一个区间中,
所以我们查询接下来的点的dfs序是不是在一个区间中即可(即计算dfs序的最大
值与最小值)。

可以发现,如果满足上面两个条件,那么深度是否加一接下来都是有合法解的。

因为如果深度加一有合法解,那么我们可以把当前点和它的子树整体挪移到上一
个点的旁边,接在上一个点的父亲的下面,这样就对应着一种深度不加一的方案。

如果深度不加一有合法解,显然从当前点开始的接下来的点一定都会接在上一个
点的父亲的子树中,因为接下来的点的dfs序一定在一个区间中。所以我们可以
把当前点接到上一个点下面,剩下的点依次挪移下来,就对应了一个深度加一的
方案。

而这两种中一定有一个有合法解,那么就可以推出另外一个也有合法解。

那么概率是多少呢?我们注意到一个点能不能深度加一与之前的决策一点关系都
没有,所以如果深度能够加一也能够不加一,那么概率分别是$1/2$。

最后,如果当前点的dfs序等于上一个点的dfs序加一,但是不满足上面的两个条
件,那么显然就是规定操作,只能不加一了。

综上所述,首先判断当前点的dfs序是否小于上一个点的dfs序,如果是的话答案
加$1$。如果当前点的dfs序为上一个点的dfs序加一,并且满足上面那两个条件,
那么答案加$1/2$。

时间复杂度为$O(n)$。

\section{小Q的修炼}

提交答案题,未做。

\section{矩阵游戏}

使用十进制快速幂,推出一行的线性表示,然后得出两行之间第一个数的递推关
系,然后求出最后一行的第一个数,然后再求出最后一个数即可。

\section{书法家}

裸dp,代码量巨大。

\section{快餐店}

枚举哪条边断环,然后就变成求直径。所有情况取最小值。

那么先求出直径在子树中的情况,断环对它没有影响。

然后把环拆成两倍的序列,然后枚举断的边,然后线段树维护即可。

\end{document}
