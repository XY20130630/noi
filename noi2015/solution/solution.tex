\documentclass[a4paper]{article}
\usepackage{ctex}
\usepackage{xeCJK}
\usepackage{amsmath}
\usepackage{amsfonts}
\usepackage{amssymb}
\usepackage{graphicx}
\usepackage{colortbl}
\usepackage{fancyvrb}
\usepackage{longtable}
\usepackage{xcolor}
\usepackage[hidelinks]{hyperref}
\usepackage[affil-it]{authblk}
\usepackage[top = 1.0in, bottom = 1.0in, left = 1.0in, right = 1.0in]{geometry}
\usepackage{amsthm}

\setCJKfamilyfont{kai}{KaiTi_GB2312}
\newcommand{\kai}{\CJKfamily{kai}}

\setCJKfamilyfont{song}{SimSun}
\newcommand{\song}{\CJKfamily{song}}

\newcommand\spc{\vspace{6pt}}
\newcommand{\floor}[1]{\lfloor {#1} \rfloor}
\newcommand{\ceil}[1]{\lceil {#1} \rceil}
\newcommand*\chem[1]{\ensuremath{\mathrm{#1}}}

\newtheorem{theorem}{Theorem}[section]
\newtheorem{lemma}[theorem]{Lemma}
\newtheorem{problem}{例题}

% \kai

\date{\today}
%\date{\yestoday}
\title{$\rm NOI2015$ 解题报告}
\author{$\mathcal Pyh$}

\begin{document}

\maketitle

\kai

\section{程序自动分析}

注意到不等判定没有传递性,于是先把相等判定进行并查集,最后再进行不等操
作。

标号比较大可以hash。

复杂度为$O(n\alpha(n))$。

\section{软件包管理器}

直接用树链剖分维护即可。

\section{寿司晚宴}

对于只有$\leq \sqrt{n}$的质因子的寿司,可以通过状压dp解决。

对于有$> \sqrt{n}$的质因子的寿司,其中一定只有一个$> \sqrt{n}$的质因子。

于是我们把所有$> \sqrt{n}$的质因子拿出来,那么它们组成的数只能一个人选
或者不选,于是就可以dp了。

\section{荷马史诗}

注意题目可以抽象为每次合并k个的合并果子。

但是每次可以合并不足k个,所以我们可以添加若干个虚点把$n-1$补足到$k-1$
的倍数,然后用堆解决即可。

\section{品酒大会}

先构出后缀数组,然后按照height数组从大到小合并集合,每次合并的时候计算
贡献即可。

\section{小园丁与老司机}

前两问直接dp,第三问就是裸的上下界最小流。

\end{document}
