\documentclass[a4paper]{article}
\usepackage{ctex}
\usepackage{xeCJK}
\usepackage{amsmath}
\usepackage{amsfonts}
\usepackage{amssymb}
\usepackage{graphicx}
\usepackage{colortbl}
\usepackage{fancyvrb}
\usepackage{longtable}
\usepackage{xcolor}
\usepackage[hidelinks]{hyperref}
\usepackage[affil-it]{authblk}
\usepackage[top = 1.0in, bottom = 1.0in, left = 1.0in, right = 1.0in]{geometry}
\usepackage{amsthm}

\setCJKfamilyfont{kai}{KaiTi_GB2312}
\newcommand{\kai}{\CJKfamily{kai}}

\setCJKfamilyfont{song}{SimSun}
\newcommand{\song}{\CJKfamily{song}}

\newcommand\spc{\vspace{6pt}}
\newcommand{\floor}[1]{\lfloor {#1} \rfloor}
\newcommand{\ceil}[1]{\lceil {#1} \rceil}
\newcommand*\chem[1]{\ensuremath{\mathrm{#1}}}

\newtheorem{theorem}{Theorem}[section]
\newtheorem{lemma}[theorem]{Lemma}
\newtheorem{problem}{例题}

% \kai

\date{\today}
%\date{\yestoday}
\title{$\rm NOI2016$ 解题报告}
\author{$\mathcal Pyh$}

\begin{document}

\maketitle

\kai

\section{优秀的拆分}

我们只需要求出每一个位置往前有多少个形如AA的字符串,每一个位置往后有多
少个形如AA的字符串,然后在相接的位置计算贡献即可。

那么我们枚举形如AA的字符串的A的长度$l$。

对于一个$l$,我们隔$l$在原串上设置一个关键点,然后第一个A一定会跨过且
仅跨过一个关键点。并且第一个A的右侧与下一个关键点的距离一定等于它的左
侧与它跨过的关键点的距离。

因此,我们可以对于相邻的两个关键点,求出它们向前、后最多能匹配的长度,
然后用差分做区间加法即可。

如果用后缀数组的话复杂度就是$O(n\log n)$。

\section{网格}

可以发现答案一定不超过2。

无解的条件为,跳蚤的个数不超过1,或者只有两只跳蚤且相邻。

答案为0的条件是,每一个蛐蛐的八联通块,周围的一圈跳蚤都连通。

答案为1的条件是,对于每一个蛐蛐周围的24个点的跳蚤都建一个点,然后连边,
求出割点,如果存在并且这个割点周围的8个点中有蛐蛐,那么就可行。

否则答案为2。

\section{循环之美}

我们只统计$gcd(x,y)=1$的数对$(x,y)$。

可以发现,$(x,y)$满足条件,当且仅当存在$a,B,l$满足,

\begin{align*}
  \frac{x\bmod y}{y}&=\frac{a}{B^l}+\frac{a}{B^{2l}}+...\\
                    &=\frac{a}{B^l-1}\\
\end{align*}

即$x_0(B^l-1)=ay$,可得$y|x_0(B^l-1)$。

因为$gcd(x_0,y)=1$,所以$y|(B^l-1)$。

那么存在$k$,满足$ky=B^l-1$,可知$gcd(ky,B^l)=1$,所以$gcd(y,B)=1$。

所以答案就为
\begin{align*}
  &\sum_{i=1}^n\sum_{j=1}^m[gcd(i,j)=1][gcd(j,B)=1]\\
  =&\sum_{d=1}^{min}\mu(d)\lfloor\frac{n}{d}\rfloor\sum_{d|j}^{m}[gcd(j,B)=1]\\
  =&\sum_{d=1}^{min}\mu(d)[gcd(d,B)=1]\lfloor\frac{n}{d}\rfloor\sum_{j=1}^{\lfloor\frac{m}{d}\rfloor}[gcd(j,B)=1]\\
\end{align*}

下底函数分块,问题转化成求

$$\sum_{i=1}^{\lfloor\frac{n}{x}\rfloor}[gcd(i,B)=1]\mu(i)$$

和求

$$\sum_{i=1}^{\lfloor\frac{n}{x}\rfloor}[gcd(i,B)=1]$$

设$g(x,y)$为$1-x$中与$B$的前$y$个质因子都互质的数的$\mu(x)$之和,那
么$$g(x,y)=g(x,y-1)+g(x/p_x,y)$$

设$f(x,y)$为$1-x$中与$B$的前$y$个质因子都互质的数的$1$之和,那
么$$f(x,y)=f(x,y-1)-f(x/p_x,y-1)$$

边界条件是杜教筛可以解决的。

\section{区间}

直接双指针扫描,区间加法维护全局最大值即可,复杂度为$O(n\log n)$。

\section{国王饮水记}

有10个结论,都知道了之后就可以斜率优化进行dp了。

\section{旷野大计算}

提交答案题。

\end{document}
