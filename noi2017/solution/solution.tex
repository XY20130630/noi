\documentclass[a4paper]{article}
\usepackage{ctex}
\usepackage{xeCJK}
\usepackage{amsmath}
\usepackage{amsfonts}
\usepackage{amssymb}
\usepackage{graphicx}
\usepackage{colortbl}
\usepackage{fancyvrb}
\usepackage{longtable}
\usepackage{xcolor}
\usepackage[hidelinks]{hyperref}
\usepackage[affil-it]{authblk}
\usepackage[top = 1.0in, bottom = 1.0in, left = 1.0in, right = 1.0in]{geometry}
\usepackage{amsthm}

\setCJKfamilyfont{kai}{KaiTi_GB2312}
\newcommand{\kai}{\CJKfamily{kai}}

\setCJKfamilyfont{song}{SimSun}
\newcommand{\song}{\CJKfamily{song}}

\newcommand\spc{\vspace{6pt}}
\newcommand{\floor}[1]{\lfloor {#1} \rfloor}
\newcommand{\ceil}[1]{\lceil {#1} \rceil}
\newcommand*\chem[1]{\ensuremath{\mathrm{#1}}}

\newtheorem{theorem}{Theorem}[section]
\newtheorem{lemma}[theorem]{Lemma}
\newtheorem{problem}{例题}

% \kai

\date{\today}
%\date{\yestoday}
\title{$\rm NOI2017$ 解题报告}
\author{$\mathcal Pyh$}

\begin{document}

\maketitle

\kai

\section{整数}

把数字按照30位一分块,把每一个块建成线段树。

对于修改,把加上/减去的数字拆成两个对不同块的贡献。这样暴力加减之后只
有对下一个块的1/-1的贡献,于是线段树一个节点维护区间是不是为全1或者全0,
找到右边第一个可以进位的位置,左边区间覆盖即可。

时间复杂度为$O(n\log n)$。

\section{蚯蚓排队}

注意到有贡献的字符串不多,全部hash起来就可以了。

\section{泳池}

首先,恰好等于$k$的问题可以通过差分转化成$\leq k$的问题。

可以发现,每一列的贡献只与最低的危险区域相关。我们可以定义这个最低的危
险区域的高度为这一列的高度。

考虑最裸的求最大子矩形的算法,即对于每一列,求出能往左右扩展的最大长度,
然后打擂台更新答案。

注意一列的左右扩展的区域被左右第一个高度小于它的列所控制,所以我们枚举
高度最小(高度相同则最左边)的列的高度,那么它的左右的列就将互不影响,
为独立的子问题。

我们设$f[i][j]$表示$i$列的泳池,高度最小的列的高度为$j$的概率(暂时不计算
高度小于$j$的格子安全的概率)。那么我们枚举这是哪一列,然后枚举左右两
边高度最小的列的g高度进行转移,即$$f[i][j]=[i\cdot (j-1)\leq k](1-{\rm p})\sum_{k=1}^i\left(\sum_{q\geq
  j}f[k-1][q]\cdot {\rm p}^{(k-1)\cdot (q-j)}\right)\left(\sum_{q\geq
  j}f[i-k][q]\cdot {\rm p}^{(i-k)\cdot (q-j)}\right)$$

显然可以前缀和优化,注意到满足$i\cdot (j-1)\leq k$的$(i,j)$只有$O(n+k\log
k)$个,这样就可以做到$O(n^2\log n)$。

可以发现,当$j\geq 2$的时候,满足条件的$(i,j)$只有$O(k\log k)$个,所以
我们先只计算$j\geq 2$的dp。

然后设$h[n]$为$n$列的泳池满足最大子矩形$\leq k$的概率。

转移的话,我们可以枚举从第$n$列向前,每一列的高度都大于1的,连续的列的
数量,然后从那一列前面进行转移。

写出转移方程之后发现是一个$k$阶线性递推,于是可以用特征多项式优化到
$O(k^2\log n)$。

于是总复杂度为$O(k^2(\log n+\log k))$。

\section{游戏}

对于每一个$x$,爆枚它是不是选$A$,然后就可以转化成2-sat问题,复杂度为
$O(2^d\cdot (n+m))$。

\section{蔬菜}

显然可以贪心,按照收益从大到小枚举每一种蔬菜,然后尽可能(在未变质的条
件下)尽可能往后放,可以用并查集维护一个位置往左第一个没满的位置在哪里。
至于额外贡献,可以从每一种蔬菜中拆出一个来做为新的蔬菜,收益为原本的收
益加上额外贡献,变质时间为原蔬菜最后变质的时间。

\section{分身术}

不会做。

\end{document}
